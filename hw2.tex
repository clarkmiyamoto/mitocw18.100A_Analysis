%%% Document Formatting
\documentclass[12pt,fleqn]{article}
\usepackage[a4paper,
            bindingoffset=0.2in,
            left=0.75in,
            right=0.75in,
            top=0.8in,
            bottom=0.8in,
            footskip=.25in]{geometry}
\setlength\parindent{0pt} % No indent

%%% Imports
% Mathematics
\usepackage{amsmath} % Math formatting
\numberwithin{equation}{section} % Number equation per section
\DeclareMathOperator{\Tr}{Tr}

\newtheorem{theorem}{Theorem}
\newtheorem{definition}{Definition}
\newtheorem{proof}{Proof}

\usepackage{amsfonts} % Math fonts
\usepackage{amssymb} % Math symbols
\usepackage{mathtools} % Math etc.
\usepackage{slashed} % Dirac slash notation
\usepackage{cancel} % Cancels to zero
\usepackage{empheq}

% Visualization
\usepackage{graphicx} % for including images
\graphicspath{ {} } % Path to graphics folder

%%% Formating
\usepackage{hyperref} % Hyperlinks
\hypersetup{
    colorlinks=true,
    linkcolor=blue,
    filecolor=magenta,      
    urlcolor=cyan,
    pdftitle={Overleaf Example},
    pdfpagemode=FullScreen,
    }
\urlstyle{same}

\usepackage{mdframed} % Framed Enviroments
\newmdenv[ 
  topline=false,
  bottomline=false,
  skipabove=\topsep,
  skipbelow=\topsep
]{sidework} %% Side-work

\usepackage{lipsum} % Lorem Ipsum example text

%%%%% ------------------ %%%%%
%%% Title
\title{Homework 2}
\author{cm6627@nyu.edu}
\date{}

%%% Contents of Document
\begin{document}
\maketitle

\section*{Problem 1}
\textbf{Assumptions:} Let $F$ be an ordered field, and $x,y,z \in F$. Assume $x < 0$ and $y < z$.

\textbf{WTS:}
$xy > xz$
\\

\textbf{Proof:} If $x < 0$, we can rewrite $x = - \tilde x$ s.t. $\tilde x > 0$. So starting w/ the inequality between $y < z$:
\begin{align}
	y & < z\\
	\tilde x y & < \tilde x 	z\\
	- \tilde x y & > - \tilde x z\\
	x y & > x z
\end{align}
QED.

\section*{Problem 2}
\textbf{Assumptions:} Let $S$ be an ordered set. Let $A \subset S$ s.t. $A$ is nonempty and finite.

\textbf{WTF:} $A$ is bounded.
\\

\textbf{Proof:} If $A$ is finite, this means $|A| = C$ s.t. $C \in \mathbb N$. Recalling this notation, this implies there exists a bijection $f: A \to \{1,2,...,C\}$. What will follow is a proof via induction, where our inductive steps occurs on $C$.\\

\underline{(Base Case)} Let $C=1$.

This means the bijection $f: A \to \{1\}$. A bijection is an invertible function, so this implies there is only 1 element in $A$. 

 
 
\section*{Problem 3 [1.1.5]}
\textbf{Assumptions:} Let $S$ be an ordered set. Let $A \subset S$ and suppose $b$ is an upper bound for $A$. Suppose $b \in A$.

\textbf{WTS:} $b = \sup A$\\

\textbf{Proof:} If $b$ is an upper bound for $A$, then $\forall a \in A, a \leq b$. But if $b \in A$, this mean $b$ is the largest element in $A$ (as there can be only 1 largest element in an ordered set). All other element in $S$ fail the definition of upper bound. Hence $b = \sup A$.\\

QED

\begin{sidework}
	To show there's only 1 largest element in an ordered set. Assume there are $b_1, b_2, ..., b_n$ largest numbers, we'll show $b_1 = b_2 = ... = b_n$\\
	
	\underline{(Base case) $n=2$:}
	
	Assuming there are two largest numbers $b_1, b_2$. However this means $b_1 \geq b_2$ and $b_2 \geq b_1$ $\implies b_1 = b_2$. So the two largest numbers are the same.
	\\
	
	\underline{(Inductive step) $n:= m$:}
	Assume $b_1 = b_2 = ... = b_n := b$. Now show $b_1 = b_2 = ... = b_n = b_{n+1}$.\\
	Since $b$ and $b_{n+1}$ both claim they're the largest number, it once again follows $b \geq b_{n+1}$ and $b_{n+1} \geq b$.  \\
	\\
	
	Hence there can only be 1 largest element in an ordered set. QED.
\end{sidework}

\section*{Problem 4 [1.1.6]}
\textbf{Assumptions:} Let $S$ be an ordered set. Let $A \subset S$ be nonempty and bounded above. Suppose $\exists \sup A \notin A$ \\
\textbf{WTS:} $A$ contains a countably infinite subset. Restated: $\exists B \subset A$ s.t. $|B| = |\mathbb N|$




\section*{Problem 5 [1.2.7]} Prove for $x,y \in \mathbb R_{>0}$.
\begin{align}
	\sqrt {xy} \leq \dfrac{x + y}{2	}
\end{align}
\textbf{Proof:}
Consider the quantity $(x-y)^2$
\begin{align}
 	0 & \leq (x - y)^2   & \forall r \in \mathbb R, r^2 \geq 0\\
 	 &  = x^2 - 2 xy + y^2 & \text{Expanded}\\
   4 xy  & \leq x^2 + 2 xy + y^2 & \text{if } r > 0 \text{ and }  
\end{align}

















 



\end{document}